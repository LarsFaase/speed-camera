\chapter{\REVIEW{Introduction}}
\label{chap:introduction}

This document is written to provide the research required to build a hidden speed camera.
The question that this research paper will try to answer is: "How can we make a speed camera that logs the speed and number plate of road users?"
To answer this research question there needs to be research to several topics such as ALPR (Automatic License Plate Recognition) and ways to detect the speed of vehicles.

\section{Background and context}

Joan Schrasser lives on a road with a 30 km/h speed limit. However, most drivers do not follow this, and so dangerous situation arise.
Children have to cross the road to go to school, and there is a bicycle lane right next to this road.
Accidents have already happened, but the municipality and local polic point at each other and don't want to do anything about it.
\cite{avans:assignment}

To show the severity of this situation, a hidden speed camera should be designed and built, so that the data can be shown to the police.


\section{Definitions and Abbreviations}

\subsection{Definitions}

Text markings

\begin{tabularx}{\textwidth}{p{2.5cm}X}
    \TODO{Marked text} & Text needs to be changed or completed.\\
    \NEW{Marked text} & Text has changed compared to the previous release.\\
    \REVIEW{Marked text} & Text is intended for review.\\
\end{tabularx}

Terminology

\begin{tabularx}{\textwidth}{p{2.5cm}X}
    D-PHY & A physical layer developed by MIPI for high-performance, cost-optimized cameras and displays.\\
    MIPI  & The MIPI Alliance (Mobile Industry Processor Interface) is a business alliance that develops specifications for the mobile ecosystem.\\
\end{tabularx}

\subsection{Abbreviations}
\begin{tabularx}{\textwidth}{lX}
    ALPR  & Automatic License Plate Recognition\\
    CPU   & Central Processing Unit\\
    CSI   & Camera Serial Interface\\
    FLOPS & Floating point operations per second\\
    GPU   & Graphics Processing Unit\\
    LPDDR & Low-Power Double Data Rate\\
    RPI   & Raspberry Pi\\
    SoC   & System on a chip\\
    WLAN  & Wireless Local Area Network\\
    HOG   & Histogram of oriented gradients\\
    SVM   & Support Vector Machines\\
    NMS   & Non-Maximum Suppression\\
    YOLO  & You Only Look Once\\
    IOU   & Intersection over Union\\
    SSD   & Single Shot MultiBox Detector\\
\end{tabularx}
